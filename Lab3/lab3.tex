\documentclass[journal]{IEEEtran}

\usepackage[spanish]{babel}
\usepackage[utf8]{inputenc}
\usepackage[T1]{fontenc}
\usepackage{graphicx}
\usepackage{amssymb}
\usepackage{amsmath}
\usepackage{amsthm}
\usepackage{booktabs}
\usepackage{gensymb}
\usepackage{stfloats}
\usepackage{float}
\usepackage{nccmath}
\usepackage{caption}
\usepackage{url}

\title{\textbf{SWITCHES}}

\author{Redes LAN \\
	\textit{Profesor:} Luis Fernando Díaz Cadavid\\ 
	\textit{Monitores:} Juan José Jaramillo Granada - 814034 \\
	Kevin Leonardo Cerpa Campanella - 814017 \\
	Universidad Nacional de Colombia - Sede Manizales}

\date{}

\begin{document}
\maketitle

\section{Descripción}
En esta práctica se introducirán los conocimientos necesarios para configurar un Switch correctamente en packet tracer, para así poderlo realizarlo físicamente.

\section{Hablando de Protocolos}
Un protocolo es
	\subsection{TELNET}
	El protocolo \textit{TELNET} es un protocolo que perimite simular el receptor como una máquina conectada a la misma red para acceder a ella y manejarla remotamente a través de una IP.
	\subsection{SSH}
	El protocolo \textit{SSH (Secure SHel)} proporciona	un inicio de sesión remoto similar al Telnet, excepto que utiliza servicios de red más seguros. El SSH proporcionaautenticación de contraseña más potente que Telnet y usa encriptación cuando transporta datos de una sesión. 	De esta manera se mantienen en privado la ID del usuario, la contraseña y los detalles de la sesión de administración. Se recomienda utilizar el protocolo SSH en lugar del Telnet, siempre que sea posible.
	\subsection{TCP/IP}
	El protocolo \textit{TCP/IP} consiste en

\section{¿Por qué se debe configurar un dispositivo?}
En escencia es por seguridad, ya que equipos como Routers, Switches, al ser dispositivos tan importantes para la comunicación en una red no cualquier persona debe acceder a ellos, ya que puede hacer un cambio, puede adquirir datos de la red de la compañía, puede restringir el acceso y un ejemplo muy claro de esto es un banco o un agente de control del estado. \\
Fuera de esto también se pueden realizar cambios para mejorar la red, a través de sus puertos, de las ventajas que nos brindan, y para esto debemos conocer como funciona y cuales son su utilidad

\section{SWITCH}
Un Switch que traducido al español para el área de red es un \textit{Conmutador} lo cual es un dispositivo cuya función es añadir funcionalidades, mejoras, control a una red LAN, además de interconectar los dispositivos que hagan parte de esta red.
Un Switche posee diversos puertos, tanto fisicos como lógicos...

\section{CISCO 2960}
Descripción de la referencia con foto y  configuración del switch en packet tracer
	
\end{document}