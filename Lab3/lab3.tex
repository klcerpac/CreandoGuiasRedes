\documentclass[journal]{IEEEtran}

\usepackage[spanish]{babel}
\usepackage[utf8]{inputenc}
\usepackage[T1]{fontenc}
\usepackage{graphicx}
\usepackage{amssymb}
\usepackage{amsmath}
\usepackage{amsthm}
\usepackage{booktabs}
\usepackage{gensymb}
\usepackage{stfloats}
\usepackage{float}
\usepackage{nccmath}
\usepackage{caption}
\usepackage{url}

\title{\textbf{Packet Tracer y eNSP}}

\author{Redes LAN \\
	\textit{Profesor:} Luis Fernando Díaz Cadavid\\ 
	\textit{Monitores:} Juan José Jaramillo Granada - 814034 \\
	Kevin Leonardo Cerpa Campanella - 814017 \\
	Universidad Nacional de Colombia - Sede Manizales}

\date{}

\begin{document}
\maketitle

\section{Descripción}
En esta práctica se introducirán los conocimientos necesarios para configurar un Switch correctamente en packet tracer, para así poderlo realizarlo físicamente.

\section{Hablando de Protocolos}
Un protocolo es
	\subsection{TELNET}
	El protocolo \textit{TELNET} consiste en
	\subsection{SSH}
	El protocolo \textit{SSH} consiste en
	\subsection{TCP/IP}
	El protocolo \textit{TCP/IP} consiste en
	
\section{Configuración de Dispositivos en Packet Tracer}
Pantallazo del doble click de un dispositivo y modo de inicio para el switch 2960

\section{CISCO 2960}
Descripción de la referencia con foto y  configuración del switch en packet tracer
	
\end{document}